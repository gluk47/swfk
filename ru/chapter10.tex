% ch10.tex
% This work is licensed under the Creative Commons Attribution-Noncommercial-Share Alike 3.0 New Zealand License.
% To view a copy of this license, visit http://creativecommons.org/licenses/by-nc-sa/3.0/nz
% or send a letter to Creative Commons, 171 Second Street, Suite 300, San Francisco, California, 94105, USA.


\chapter{И что теперь?}

Поздравляю тебя! Ты добрался до конца.

Я надеюсь, что смогла тебя научить тем важным деталям, из которых строится большинство программ, написанных на популярных языках программирования\footnote{Хотя некоторые языки программирования очень сильно отличаются от Питона, и программы в них пишутся совсем не так. Например, такие другие — функциональные языки, ещё логические бывают.}. Эти знания помогут тебе как научиться программировать по-настоящему, так и овладеть другими языками. Ни один из языков не подходит идеально для решения всех задач: что-то проще делать на одном языке, что-то другое — на другом. Именно поэтому их так много и придумали: каждый для своих задач. Ну и ещё разным людям просто нравятся разные языки, так что советую попробовать всякие.

Вот например, есть простой язык, чтобы попробовать программирование игр: BlitzBasic («Блиц Бэйсик») (\url{http://www.blitzbasic.com}), основанный на языке BASIC («Бэйсик»), придуманном для обучения. А игры в интернете часто пишут на языке Flash («Флэш»).

%TODO перевести вот это всё и оценить годность рекомендаций: современность и наличие русского перевода.
%
%If you're interested in programming Flash games, possibly a good place to start would be `Beginning Flash Games Programming for Dummies', a book written by Andy Harris, or a more advanced reference such as `The Flash 8 Game Developing Handbook' by Serge Melnikov.  Searching for `flash games' on \href{http://www.amazon.com}{www.amazon.com} will find a number of books on this subject.
%
%Some other games programming books are: `Beginner's Guide to DarkBASIC Game Programming' by Jonathon S Harbour (also using the Basic programming language), and `Game Programming for Teens' by Maneesh Sethi (using BlitzBasic). Be aware that BlitzBasic, DarkBasic and Flash (at least the development tools) all cost money (unlike Python), so Mum or Dad will have to get involved before you can even get started.
%
%If you want to stick to Python for games programming, a couple of places to look are: \href{http://www.pygame.org}{www.pygame.org}, and the book `Game Programming With Python' by Sean Riley.
%
%If you're not specifically interested in games programming, but do want to learn more about Python (more advanced programming topics), then take a look at `Dive into Python' by Mark Pilgrim (\href{http://www.diveintopython.org}{www.diveintopython.org}).  There's also a free tutorial for Python available at: \href{http://docs.python.org/tut/tut.html}{http://docs.python.org/tut/tut.html}.  There's a whole pile of topics we haven't covered in this basic introduction so, at least from the Python perspective, there's still a lot for you to learn and play with.

Если же ты хочешь писать страницы в интернете и целые сайты, то тебе пригодятся языки HTML, Javascript и CSS, а ещё (потом, для целых сайтов) — PHP. Писать простые страницы очень легко: нужно написать файл на языке HTML и потом открыть его браузером (таким как Mozilla Firefox, например).

Чтобы писать драйверы, операционные системы и производительные программы обычно используют C, C++.

Чтобы писать научные статьи и просто автоматически красиво отформатированный текст, есть языки \TeX и его версии \LaTeX («Латех»), \XeLaTeX. Последний (который \XeLaTeX) был использован, чтобы писать эту книгу. Нужно было написать компьютеру, какой текст, картинки и главы мне нужны, а он сам расположил это на бумаге, расставил переносы, сделал нужные размеры шрифтов, вставил картинки в нужные места.

А ещё бывают языки (Verilog и VHDL), на которых компьютеру описывают, что должно быть в микросхеме, которую хочется сделать, а он отвечает, как нужно все детали расположить, чтобы всё работало побыстрее и не перегревалось.

Что выбрать дальше — решать тебе; помимо перечисленных здесь языков есть много других вариантов, каждый из которых удобен в чём-то своём.

\emph{Удачи тебе и не забывай, что программированием надо наслаждаться, а иначе нету смысла.}

\newpage