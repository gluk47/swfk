% preface.tex
% This work is licensed under the Creative Commons Attribution-Noncommercial-Share Alike 3.0 New Zealand License.
% To view a copy of this license, visit http://creativecommons.org/licenses/by-nc-sa/3.0/nz
% or send a letter to Creative Commons, 171 Second Street, Suite 300, San Francisco, California, 94105, USA.


\chapter*{Вступление}\normalsize
    \addcontentsline{toc}{chapter}{Вступление}
\begin{center}
{\em Пара слов для родителей...}
\end{center}
\pagestyle{plain}

Уважаемый родитель или иной управляющий компьютером!

Чтобы ваш ребёнок смог начать знакомиться с программированием, вам нужно установить Python на компьютер. Эта книга была недавно обновлена до версии Python 3.0, самой новой и несовместимой с предыдущими, так что если у вас установлена более старая версия Pyhton, вам стоит скачать и более старую версию этой книги.

Установка Python — достаточно простая задача, но есть несколько тонкостей — в зависимости от используемой операционной системы. Если вы только что купили сверкающий новый компьютер и не имеете никаких идей, что с ним делать, а предыдущее предложение начало вызывать у вас нервную дрожь или холодный пот, то, пожалуй, лучше вам найти кого-то кто сделает это за вас. Установка Python может занять от 15 минут до пары часов в зависимости от скорости интернета и компьютера.

Прежде всего, скачайте и установите последнюю версию Python 3 для вашего дистрибутива. Дистрибутивов очень много, так что инструкции для всех тут привести не получится… да и скорее всего, если вы используете Linux, то уже знаете, как это сделать. Наверное, вы даже возмущены самой идеей того чтобы рассказывать вам, как что-либо устанавливать, так что тут я остановлюсь.

\note{После установки…}

…Вам, возможно, придётся в течение первых пары глав посидеть с ребёнком рядом, но после нескольких примеров ему будет только приятнее читать книгу самому (с компьютером вместе). Нужно рассказать ребёнку, как вводить команды в консоль, как пользоваться текстовым редактором (наподобие блокнота; Microsoft Word никак не подойдёт), открывать и сохранять файлы в этом редакторе. Всё остальное расскажет эта книга.

\vspace{12pt}
\noindent
Спасибо за уделённое время; с наилучшими пожеланиями,\\
КНИГА.
